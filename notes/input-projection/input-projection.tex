\documentclass{article}

\usepackage[utf8]{inputenc}
\usepackage{parskip}
\usepackage{amssymb,amsfonts,amsmath,amscd}
\usepackage{bm}
\usepackage{hyperref}
\usepackage[pdftex]{graphicx}
\usepackage{url}
\usepackage[usenames,dvipsnames]{color}
\usepackage{enumitem}
\usepackage{mathtools}
\usepackage{float}

% for typesetting in-text numbers and units
\usepackage{siunitx}

% for checking internal references
% \usepackage{refcheck}

% improved typesetting
%\usepackage{microtype}

% Table stuff.
\usepackage{booktabs}
\usepackage{makecell}
\usepackage{multirow}

% Useful commands.
% \input{commands}

\title{OCS2 Input Projection}
\author{Adam Heins}

\begin{document}

\maketitle

OCS2 has the option to use input projection to resolve state-input equality
constraints. Consider an equality constraint (or linear approximation thereof)
of the form
\begin{equation}\label{eq:equality_constraint}
  \bm{C}\bm{x} + \bm{D}\bm{u} + \bm{e} = \bm{0}.
\end{equation}
We would like to find an alternative input~$\tilde{\bm{u}}$ where
\begin{equation}\label{eq:projection}
  \bm{u} = \bm{P}_x\bm{x}+\bm{P}_u\tilde{\bm{u}} + \bm{p}_0,
\end{equation}
such that~\eqref{eq:equality_constraint} is always satisfied.
Substitute~\eqref{eq:projection} into~\eqref{eq:equality_constraint} to obtain
\begin{equation*}
  (\bm{C}+\bm{D}\bm{P}_x)\bm{x} + \bm{D}\bm{P}_u\tilde{\bm{u}} + \bm{D}\bm{p}_0 + \bm{e} = \bm{0},
\end{equation*}
from which we obtain the relationships
\begin{align*}
  \bm{P}_u &= \mathrm{Null}(\bm{D}), & \bm{P}_x &= -\bm{D}^{-1}\bm{C}, & \bm{p}_0 &= -\bm{D}^{-1}\bm{e}.
\end{align*}
From these results it is clear that~$\bm{D}$ \emph{must have full row rank!}

Now suppose we have some other function
\begin{equation}\label{eq:other_func}
  \bm{f} = \bm{f}_0 + \bm{F}_x\bm{x} + \bm{F}_u\bm{u}
\end{equation}
to which we'd like to apply the projection~\eqref{eq:projection}.
Substitute~\eqref{eq:projection} into~\eqref{eq:other_func} to get
\begin{equation*}
  \bm{f}(\bm{x},\bm{u}) = \underbrace{\bm{f}_0 + \bm{F}_u\bm{p}_0}_{\tilde{\bm{f}}_0} + \underbrace{(\bm{F}_x + \bm{F}_u\bm{P}_x)}_{\tilde{\bm{F}}_x}\bm{x} + \underbrace{\bm{F}_u\bm{P}_u}_{\tilde{\bm{F}}_u}\tilde{\bm{u}}
\end{equation*}
and take the function under the change-of-variables to be
\begin{equation*}
  \tilde{\bm{f}}(\bm{x},\tilde{\bm{u}}) = \tilde{\bm{f}}_0 + \tilde{\bm{F}}_x\bm{x} + \tilde{\bm{F}}_u\tilde{\bm{u}}.
\end{equation*}

\end{document}
